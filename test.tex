\documentclass[11pt]{article}
\usepackage{cite}

\begin{document}

\title{The current state}
\author{All of us}
\date{Today}
\maketitle

\section{Dulce's work}

\begin{enumerate}
\item change the meassured value to the bitscore.
\item only take into acount the coordenates for gene, and not the combined coordenates.
\item Test on real data could be made on the next week
\end{enumerate}

\section{Observability}

\begin{enumerate}
\item Stop asking this.
\item Well if you ask aboutit, defined as only one theory
\item Fitch relation is what can be obtained for HGT
\item Marc has done some work obout it in which both orthology and fitch relations are considered.
\item Nothing to be done hier
\end{enumerate}


\section{rBMG BMG ... OMG}

\begin{enumerate}
\item only bidirectional edges give orthology information
\item is not clear if the resulting relations can be a cograph
\item the triplets do not give a uniq solution. can not remember if even a consistent one
\item There are 3 basic structures in which a tree structure can be infered, however, it doesnt have unique solutions.
\item aditional contrains migth come from consistency with species labeling
\item the resiprocal best mathc migth not give enough information, so other measurement has to be taken into account.
\item the directed edges can give more measurment information
\item Simbolic ultrametrics can give more measurment information
\item Additive distance matrix can also give more measurment information
\item The empirical part of the cographs in rBMG 
\item Given no losses, but duplication constrains, how much do we 
\end{enumerate}


% \bibliography{/home/bloodfield/WWCiencia/BibTex/Liepzig-Reconciliation}{}
% \bibliographystyle{plain}
\end{document}
