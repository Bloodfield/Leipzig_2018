\documentclass[11pt]{article}
\usepackage{latexsym}

\begin{document}

\title{The current state}
\author{All of us}
\date{Today}
\maketitle

\section{Dulce's work}

\begin{enumerate}
\item change the meassured value to the bitscore.
\item only take into acount the coordenates for gene, and not the combined coordenates.
\item Test on real data could be made on the next week
\end{enumerate}

\section{Observability}

\begin{enumerate}

\item Fitch relation is what can be obtained for HGT

The mexican mafia tried to develop observability axioms considering $ \{ \odot , \bullet , \triangle , \Box \} $. However, defining an observable spetiation was a trouble.
The question tured into what relations can be drawn between species. We found that Marc did a lot of work regarding this mather, however he doesn't take into consideration the biological contrains of the used methods to detect orthology and HGT's of a given methodology.

After talking about this in Leipzig, the conclusion was to ask what are the relations that can be observed?, the HGT as las comon ancester of two genes or the fitch relations?

Resolving this doubt, we should be able to define observability axioms.

If we define a given a model and techniques, basic observability axioms can be constructed in order to develop any forward mathematical theory.

It was concluded, after the meeting, that the Fitch reations were the ones that can be obtained by methodology.

Now we have to analize Fitch relations and orthology relations together.

\item Marc has done some work obout the escenarios when both orthology and fitch relations are considered.

However, March has already some work about it. A message was sent, and a repository is made, and we are waiting for it.

\item Stop asking this.

Since the meaning of observability is really ambigous, it creats a lot of problems comunication and axiomatic construction. Therefore, the idea of general observability can not be deffined in a congruent matter. This means again "If we define a given a model and techniques, basic observability axioms can be constructed in order to develop any forward mathematical theory".


\end{enumerate}


\section{rBMG BMG ... OMG}

\begin{enumerate}
\item only bidirectional edges give orthology information

Manuela started to work on the n-species rBMG. The main inference is that relations indicate genes related to each other at the same time. However, there seems to not be enough information in order to have a unique inferred tree. Additional constrains can be found in the species tree topology, or extra meassurements of distance between genes.

\item There are 3 basic structures in which a tree structure can be inferred

The rBMG has different recurrent structures that can be resolved independently. (imagine nice drawings here)

\item is not clear if the resulting relations can be a cograph

At this point is not clear if a rBMG gives a cograph. in one hand, the basic structures clearly have P4's. The questis here is. Is it posible to have a the same inference and a p4 free rBMG? o.0 ?

\item aditional contrains migth come from consistency with species labeling

If we considere that the labeled gene tree has to be mapable to a species tree, then aditional contrains can be aplied in order to minimize inferred losses.
This parsimony aproach can cause aditional troubles too, adding losses and relabeling inner nodes as duplications have to applied.

\item the directed edges can give more measurment information

Aditional information can be adquierd from the BMG. directed edges can be used to detect HGT. This causes to change orthology edges. Since a HGT will replace an orthology relation, a new bidirectional relation should apear.

\item Simbolic ultrametrics can give more measurment information

If turns out that rBMG acctually gives a cograph, then a direct relation with Simbolic ultrametrics can be done.
In this case the measurment is given by the succesion of inner labeld edges.
The info in the rBMG plus the análisis of the cograph, can give a new resolved tree. Whether it is unique o even possible in every scenario, are questions that follows.

\item The empirical part of the cographs in rBMG 

Of course there is also the possibility to compare how much far appart does a theoretical cotree is appart from a rBMG-tree. Artifitial gene stories can be made taking into account duplications and/or gene losses and/or HGT (any possible convination).  Then a rBMG can be created from it. And analize how much the actual orthology relations differ from the rBMG. A gut-feeling says that difference should depend from gene loss and number of duplications. We need to work on this details.


\end{enumerate}


% \bibliography{/home/bloodfield/WWCiencia/BibTex/Liepzig-Reconciliation}{}
% \bibliographystyle{plain}
\end{document}
